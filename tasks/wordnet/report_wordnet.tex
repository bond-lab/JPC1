\documentclass[12pt,a4paper]{article}

\title{JPC1: Metaphor and Metonymy in English, explored through ChainNet and Wordnet}
\author{Francis Bond \\
  \href{mailto:bond@ieee.org}{bond@ieee.org}}

%%% links
\usepackage[dvipsnames]{xcolor}
\usepackage[colorlinks=true]{hyperref}
\hypersetup{
  citecolor=OliveGreen, % a very dark green
  linkcolor=NavyBlue,
  urlcolor=NavyBlue
}
\usepackage{url}
%%% references
\usepackage{natbib}
%\usepackage[natbibapa]{apacite}
\usepackage{booktabs}
\usepackage{graphicx}


\begin{document}
\maketitle

\section{Introduction}
\label{sec:introduction}

I look at the tropes (metaphors and metonyms) identified in ChainNet \citep{maudslay-etal-2024-chainnet-structured}, and use wordnet to tell us something more about them.

\section{Background}
\label{sec:background}

We will use wordnet to quantitatively test some of the hypothesis about Metaphor and Metonymy.

\subsection*{Resources}
\label{sec:resources}

Chainnet, Princeton Wordnet 3.0 \citep{_Fellbaum:1998}, python wn module \citep{Goodman:Bond:2021}.

\section{Approach}
\label{sec:approach}

For every trope, we look at either the src and target nodes, or some difference between them.  The code is in \texttt{analyse-tropes.py}

For the nodes we look at:
\begin{itemize}
\item Depth (distance to root)
\item Synonymy density (number of synonyms)
\item Abstractness%\marginpar{ToDo}
\item Specificity\marginpar{ToDo}
\item Topic (lexicographer file)
\end{itemize}

Depth is just the maximum depth from the root (\texttt{synset.min\_depth()}).  Synonymy density is the number of synonyms the concept has (\texttt{len(ss1.lemmas())}).  Topic is the lexicographer file.  Abstractness is calculated as suggested by \citet{Mensa:Porporato:Radicioni:2018} 0 if the synset is a hyponym of \textit{\textbf{physical entity}}, 1 otherwise.

We should also calculate inclusiveness \citep[p 719]{Iliev:Axelrod:2017}.
This is defined as: if a concept $c$ has $n$ descendents (both direct
and indirect) in a tree that has a total of $N$ nodes, then:
\begin{equation}
\textnormal{inclusiveness} (c) = −log ((n+1)/N)  
\end{equation}




For the differences we look at:
\begin{itemize}
\item Distance between nodes (path)
%\item Distance between nodes (information)
\item Difference in depth between nodes
\end{itemize}



\section{Results}
\label{sec:results}

\input{build/summary_tables.tex}


\begin{figure}
  \centering

  \includegraphics{build/path_distance_comparison_errorbar.png}
  \caption{Depth Distance Comparison}
  \label{fig:depth}
\end{figure}

ToDo: Decide which to use

\includegraphics[width=\textwidth]{build/metaphor_topic_heatmap.png}
\includegraphics[width=\textwidth]{build/metonym_topic_heatmap.png}
\includegraphics[width=\textwidth]{build/synonyms_comparison.png}
\includegraphics[width=\textwidth]{build/depth_comparison_errorbar.png}
\includegraphics[width=\textwidth]{build/path_distance_comparison_errorbar.png}
\includegraphics[width=\textwidth]{build/depth_comparison.png}
\includegraphics[width=\textwidth]{build/path_distance_comparison.png}

\section{Discussion}
\label{sec:discussion}

Path depth is as expected, Abstractness less change then expected.\marginpar{Expand!}

\section{Conclusions}
\label{sec:conclusions}

ToDo



\bibliographystyle{apalike}
%\bibliography{abb,mtg,nlp,ling,local}
\bibliography{local}



\end{document}
